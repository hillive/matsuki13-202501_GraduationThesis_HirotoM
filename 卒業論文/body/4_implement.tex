\chapter{引用}

%%%%%%%%%%%%%%%%%%%%%%%%%%%%%%%%%%%%%%%%%%%%%%%%%%%%%%%%%%%%%%%%%%%%%%%
\begin{comment}
    \begin{textblock}{2}(1, 16.5)
        空行→
    \end{textblock}
    
    \begin{textblock}{2}(1, 18.5)
        字下げ→
    \end{textblock}
        
    \begin{textblock}{2}(1, 20.5)
        空行→
    \end{textblock}
    
    \begin{textblock}{11}(9, 20.5)
        ←読点までが元の文なので文献番号はその後につける
    \end{textblock}
    
    \begin{textblock}{7}(14, 26.5)
        ↑同じく読点までが元の文なので
    
        "」"と文献番号はその後につける
    \end{textblock}
\end{comment}
%%%%%%%%%%%%%%%%%%%%%%%%%%%%%%%%%%%%%%%%%%%%%%%%%%%%%%%%%%%%%%%%%%%%%%%    


\section{直接引用}

引用元の文献に書かれている文章を変更なしで引用する方法を直接引用と呼ぶ。
直接引用をする場合、(1)自分が書いた文章と区別がつくこと、(2)引用元に書かれている文を変更しないこと、(3)出典(引用した著作物の情報)、の3つが必要である。

\subsection{長めの文章の直接引用}

長め(数行)の文章を直接引用する場合、文章の前後は空行を入れ、各行頭を字下げする。
引用文の最後(句読点の後)には、その出典に該当する参考文献の番号やラベルをつける。
次の赤字部分がその例である。

% 東京工科大学学長の大山は、Webページの中で以下のように述べている。

% \begin{quotation}
%     \textcolor{red}{
%     「実学主義」教育では、実践的な専門分野の知識や技術とその基礎・原理原則を身に付けるとともに、
%     国際的な教養や豊かな人間性を養うことにより、社会の変化に柔軟に対応して活躍できる適応力を育んでいます。
%     本学ではICT教育に力を入れ、1999年よりノートPCを用いた授業を開始しました。
%     現在では全学部ノートPC必携で、教育・学修支援を実現するプラットフォーム「Moodle」を導入しています。
%     このコロナ禍においても、構築した学修環境は遠隔授業の取り組みに大いに役立てられています。\cite{学長挨拶大山恭弘23:online}}
% \end{quotation}

% 必携のノートPCは学部によって異なる。蒲田のデザイン学部ではMacが、それ以外の学部ではWindows PCを前提とした授業が進められている。

東京工科大学学長の香川は、Webページの中で以下のように述べている。

\begin{quotation}
    \textcolor{red}{
    東京工科大学は「社会の変化とそれに伴う課題を理解し、自分の力で課題解決できる人材の育成」をめざし、
    社会で求められる力をはぐくむための学習環境を整備しています。
    例としては、各分野で活躍している教員のサポート下で行う先端的研究や、
    2024年度からさらに強化される企業での就業体験、海外実習、地域連携活動などのカリキュラムが挙げられます。
    教室で学んだ知識を生かしながら新たな気づきや実践的な学びを修得することは、必ず未来の活動に生きるはずです。\cite{学長挨拶23:online}}
\end{quotation}

企業での就業体験を取り入れたカリキュラムは工学部において先行して取り入れられている。

\subsection{短めの文章の直接引用}

短い文章を直接引用する場合は該当箇所を「」で括り、それに続けて文献番号を入れる。次の赤字部分がその例である。

大山は学長挨拶の中で
\textcolor{red}{
 「また、デジタル技術とその使い方についての学修や国際人としての教養の体得は、専門分野を問わない社会人基礎力となるはずです。」\cite{学長挨拶23:online}}
% 「⼊学から就職・進学まで⼀貫したサポートで全教職員がみなさんの夢の実現を応援します。」\cite{学長挨拶大山恭弘23:online}}
と述べている。

% 引用の形式を取らずに他者の文章を自身の論文に記載することを「盗用・剽窃」と呼ぶ。
% 盗用・剽窃の対象は学術論文に限らない。ウェブサイトや過去の卒業論文も含まれる。
% 盗用・剽窃が発覚した論文は受理しないことがあるので、慎重に執筆をしてほしい。

\section{間接引用}

引用元の文献に書かれている内容を自分で要約して紹介する方法を間接引用と呼ぶ。
間接引用では字下げや「」による括りの必要はないが、参考文献の番号は必須である。
文献の文章を要約する際は、その主張が変わらないようにする。
文献番号は文末の句点の前に入れる。
次の赤字部分がその例である。

\textcolor{red}{
% 大山はICT教育に力を入れてきた本学の学修環境がコロナ禍における遠隔講義の取り組みに役立ったと述べている\cite{学長挨拶大山恭弘23:online}。
香川は大学生活が社会に出るまでの大切な準備期間であると述べている\cite{学長挨拶23:online}。
}

%%%%%%%%%%%%%%%%%%%%%%%%%%%%%%%%%%%%%%%%%%%%%%%%%%%%%%%%%%%%%%%%%%%%%%%
\begin{comment}
    \begin{textblock}{11}(9, 7.5)
        \noindent
        ↑間接引用では節末・文末の句読点の「前」に文献番号をつける
    \end{textblock}
\end{comment}
%%%%%%%%%%%%%%%%%%%%%%%%%%%%%%%%%%%%%%%%%%%%%%%%%%%%%%%%%%%%%%%%%%%%%%%


\section{参考文献の種類}

自身の主張を補うための参照先として一般的なものは学術論文\cite{neko}や国際会議論文\cite{kajiyama_shapio}である。
これ以外には書籍\cite{kinoshita}、ウェブページ\cite{LIV}、過去の卒業論文\cite{sotsuron}などが挙げられる。

巻末の参考文献一覧には、元となる文献を特定するために必要な情報を通し番号付きで列挙する。
学術論文や国際会議論文が参考文献の場合は、その著者・題目・掲載雑誌名・巻号・掲載ページ・出版年を記載する。
書籍の場合は出版社、卒業論文・修士論文・博士論文では大学名も記載する。
Webページの場合はURLの他に\underline{そのURLを閲覧した日}の記載が必須である。

このサンプルの参考文献形式(文献番号や著者などの記述形式)は \verb|junsrt.bst| である。
どの参考文献形式を使うかは研究室や研究分野によって異なるので、指導教員の指示に従うこと。